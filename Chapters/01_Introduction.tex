\chapter{Introduction}

% Write your introduction here.

\section{Background}
The \acrfull{lcs} problem is a fundamental problem in computer science, with applications in multiple domains, bioinformatics being one of the most prominent.
This is due to the fact that \gls{dna} and \gls{rna} sequences are usually encoded as strings for bioinformatics workflows, in a format called FASTA, originally described by Pearson and Lipman in 1985 \cite{pearson}.

The \acrshort{lcs} problem is defined as follows: 
\begin{displaycquote}{lcsdef}
  Given two strings S and T, each of length at most n, the longest common substring (\acrshort{lcs}) problem is to find a longest substring common to S and T.  
\end{displaycquote}
For example, given the strings "AG\textbf{CTAGC}" and "T\textbf{CTAGC}TA", the longest common substring is "CTAGC", which has a length of 5.
The \acrshort{lcs} problem can be solved using various algorithms, such as dynamic programming, suffix trees, suffix arrays, and suffix automata, each with different time and space complexities.

Of particular interest are the \acrfull{sam} (also known as \acrfull{dawg}) and the \acrfull{sa} with \acrfull{lcp} Array (also known as an \acrfull{esa}), which both yield the search result in linear time, but differ in their construction time and space requirements.

\section{Research Questions}
