\chapter{Introduction}

% Write your introduction here.

\section{Background}
The \acrfull{lcs} problem is a fundamental problem in computer science, with applications in multiple domains, bioinformatics being one of the most prominent.
This is due to the fact that \gls{dna} and \gls{rna} sequences are usually encoded as strings for bioinformatics workflows, in a format called FASTA, originally described by Pearson and Lipman in 1985 \cite{pearson}.

The \acrshort{lcs} problem is defined as follows: 
\begin{displaycquote}{lcsdef}
  Given two strings S and T, each of length at most n, the longest common substring (\acrshort{lcs}) problem is to find a longest substring common to S and T.  
\end{displaycquote}
For example, given the strings "AG\textbf{CTAGC}" and "T\textbf{CTAGC}TA", the longest common substring is "CTAGC", which has a length of 5.
The \acrshort{lcs} problem can be solved using various algorithms, such as dynamic programming, suffix trees, suffix arrays, and suffix automata, each with different time and space complexities.

Of particular interest are the \acrfull{sam} (also known as \acrfull{dawg}) and the \acrfull{sa} with \acrfull{lcp} Array (also known as an \acrfull{esa}), 
which both yield the search result in linear time, but differ in their construction time and space requirements.

Both of these data structures are derived from the Suffix Tree,, the first linear time and space data structure for solving the \acrshort{lcs} problem, 
which was introduced by Peter Weiner in 1973 \cite{weiner}.

\subsection{Suffix Tree}
A Suffix Tree (originally called a bi-tree)

\subsection{Suffix Automaton}

A \acrfull{sam} is a data structure that represents all the substrings of a given string in a compact way.

\begin{displaycquote}{samdef}
  An efficient and compact data structure for representing a full index of a set of strings is a \acrlong{sam}, a minimal
  deterministic automaton representing the set of all suffixes of a set of strings. Since a substring is a prefix of a suffix, a
  suffix automaton can be used to determine if a string \textit{x} is a substring in time linear in its length \textit{O(|x|)}, which is optimal.
\end{displaycquote}

The \acrshort{sam} was first introduced by Blumer et al. in 1985 \cite{sam}, 
and its use in solving the \acrshort{lcs} problem has been explored in various works, 
such as in the book Text Algorithms by Crochemore and Rytter, in chapter 6 \cite{crochemore1994text}.

A notable feature of the \acrlong{sam} is that it can be constructed with just one string,
and in fact, the same automaton can then be reused to find the \acrlong{lcs} of that string and any other string, 
by traversing the automaton with the second string and keeping track of the longest match found.

\subsection{Enhanced Suffix Array}

\begin{displaycquote}{esadef}
  The generic name \textit{enhanced suffix array} stands for data structures consisting of the suffix array and additional tables.
\end{displaycquote}

\section{Research Questions}
