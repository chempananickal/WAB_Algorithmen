\documentclass[12pt,a4paper,twoside,open=any]{scrreprt}

% Shared LaTeX settings
\usepackage{float}
\usepackage[T1]{fontenc}
\usepackage[utf8]{inputenc}
\usepackage[english,ngerman]{babel}
\usepackage{csquotes}
\usepackage{geometry}
\usepackage{setspace}
\usepackage{graphicx}
\usepackage{booktabs}
\usepackage{longtable}
\usepackage{hyperref}
\usepackage{tabularx}
\usepackage{makecell}
\usepackage{rotate}
\usepackage[acronym,automake]{glossaries}
% \usepackage[backend=bibtex,style=authoryear,autocite=footnote]{biblatex}
\usepackage[backend=bibtex,style=numeric,sorting=none]{biblatex}

\geometry{margin=1in}
\setstretch{1.2}
\hypersetup{colorlinks=true,linkcolor=black,citecolor=black,urlcolor=black}
\setlength{\parskip}{1em}

% Reduce space before/after chapter, section, and subsection headings
\RedeclareSectionCommand[
  beforeskip=1em,
  afterskip=1em
]{chapter}
\RedeclareSectionCommand[
  beforeskip=0.5em,
  afterskip=0.5em
]{section}
\RedeclareSectionCommand[
  beforeskip=0.5em,
  afterskip=0.5em
]{subsection}

\addbibresource{references.bib}
\makeglossaries

% Custom environment for benchmark tables
% Wraps tabularx tables without caption (won't appear in list of tables)
\newenvironment{benchmarktable}[2]{% #1=title, #2=description
    \begin{table}[H]%
        \small%
        \raggedright%
        \textbf{#1}\\%
        \textit{#2}\\[4pt]%
        \centering%
}{%
    \end{table}%
}

% Make \cite produce footnotes
% \let\cite\footcite
\let\cite\supercite
% Shared document metadata
\newcommand{\ThesisTitle}{Comparing Suffix Automata Against Suffix Arrays For Longest Common Substring Queries}
\newcommand{\ThesisSubtitle}{A Proof-of-Concept Implementation Based On A Real Use Case}
\newcommand{\AuthorName}{Rubin Chempananickal James}
\newcommand{\AuthorEmail}{rubin.chempananickal-james@stud-provadis-hochschule.de}
\newcommand{\UniversityName}{Provadis School of International Management and Technology}
\newcommand{\DepartmentName}{Information Technology}
\newcommand{\ModuleName}{Algorithmen und Datenstrukturen}
\newcommand{\ReviewerName}{Prof. Dr. Volker Scheidemann}
\newcommand{\SubmissionDate}{\today}

% Glossary entries
\newglossaryentry{dna}{name=DNA, description={Deoxyribonucleic Acid, the molecule that carries genetic information in living organisms}}
\newglossaryentry{rna}{name=RNA, description={Ribonucleic Acid, a molecule that is transcribed from DNA and which then gets translated into proteins}}
\newglossaryentry{base}{name={base}, description={The building blocks of DNA and RNA, represented by the letters A, C, G, T (for DNA) and A, C, G, U (for RNA)}}
\newglossaryentry{sentinel}{name=sentinel, description={A unique character used to separate concatenated strings in a suffix array or suffix tree, 
which is lexicographically smaller than any other character in the strings. 
Usually characters like \textbackslash0 (null character), \$, or \# are used as sentinels, as they are ahead of all other characters in the ASCII table}}
% Abbreviations
\newacronym{sam}{SAM}{Suffix Automaton}
\newacronym{dawg}{DAWG}{Directed Acyclic Word Graph}
\newacronym{lcs}{LCS}{Longest Common Substring}
\newacronym{sa}{SA}{Suffix Array}
\newacronym{lcp}{LCP}{Longest Common Prefix}
\newacronym{esa}{ESA}{Enhanced Suffix Array}

\begin{document}
\selectlanguage{english}

% Front matter in Roman numerals
\pagenumbering{Roman}

\begin{titlepage}
\thispagestyle{empty}
\begin{flushright}
\includegraphics[height=2.5cm]{Images/Cover/provadis-hochschule.pdf}
\end{flushright}
\vspace{1cm}
\centering
{\Large WAB \par}
\vspace{0.5cm}
{\Large \UniversityName \par}
\vspace{1cm}
{\LARGE \textbf{Expos\'e} \par}
\vspace{0.5cm}
{\Large \textbf{\ThesisTitle} \par}
\vspace{0.5cm}
{\Large \ThesisSubtitle \par}
\vspace{2cm}
{\large \AuthorName \par}
{\large \AuthorEmail \par}
{\large Matriculation Number: \MatriculationNumber \par}
\vspace{1.5cm}
{\large Department: \DepartmentName \par}
{\large Module: \ModuleName \par}
{\large Reviewer: \ReviewerName \par}
\vfill
{\large \SubmissionDate \par}
\end{titlepage}

\tableofcontents

% Glossary and abbreviations in front matter
\printglossary[title=Glossary]
\addcontentsline{toc}{chapter}{Glossary}
% \printglossary[type=\acronymtype,title=Abbreviations,nonumberlist]

% Exposee content in Arabic numerals
\clearpage
\chapter*{Expos\'e}
\addcontentsline{toc}{chapter}{Expos\'e}
\pagenumbering{arabic}

\section*{Introduction}
\addcontentsline{toc}{section}{Introduction}
The \acrlong{lcs} (\acrshort{lcs}) problem is a fundamental problem in computer science, with applications in multiple domains, bioinformatics being one of the most prominent.
This is due to the fact that \gls{dna} and \gls{rna} sequences are usually encoded as strings for bioinformatics workflows, in a format called FASTA, originally described by Pearson and Lipman in 1985 \cite{pearson}.

The \acrshort{lcs} problem is defined as follows: 
\begin{displaycquote}{lcsdef}
  Given two strings S and T, each of length at most n, the longest common substring (\acrshort{lcs}) problem is to find a longest substring common to S and T.  
\end{displaycquote}
For example, given the strings "AG\textbf{CTAGC}" and "T\textbf{CTAGC}TA", the longest common substring is "CTAGC", which has a length of 5.
The \acrshort{lcs} problem can be solved using various algorithms, such as dynamic programming, suffix trees, suffix arrays, and suffix automata, each with different time and space complexities.

Of particular interest are the \acrfull{sam} (also known as \acrfull{dawg}) and the \acrfull{sa} with \acrfull{lcp} Array (also known as an \acrfull{esa}), which both yield the search result in linear time, but differ in their construction time and space requirements.

\section*{Research Question}
\addcontentsline{toc}{section}{Research Question}
The primary research question of this paper is:
\begin{quote}
  How do the \acrlong{sam} and the \acrfull{esa} compare in terms of time and space complexity when solving the \acrlong{lcs} problem in an everyday programming context?
\end{quote}

Python\cite{python} was chosen as the programming language for this paper due to the fact that it is by far the most popular programming language for bioinformatics \cite{pythonproof}.

\section*{Objectives}
\addcontentsline{toc}{section}{Objectives}
The objectives of this paper are as follows:
\begin{itemize}
    \item To review the existing literature on algorithms for solving the \acrlong{lcs} problem, with a focus on the \acrlong{sam} and the \acrfull{esa}.
    \item To implement the \acrlong{sam} and the \acrfull{esa} algorithms for solving the \acrlong{lcs} problem.
    \item To evaluate the time and space complexity of both algorithms using a Python script that measures their performance on a variety of test cases.
    \item To analyze the results and draw conclusions about the trade-offs between the two approaches in terms of their efficiency and practicality for real-world applications.
\end{itemize}


\section*{Methodology}
\addcontentsline{toc}{section}{Methodology}
The \acrshort{sam} and \acrshort{esa} algorithms will be implemented in Python, and only synthetic test cases will be used to evaluate their performance.
The amount of memory consumed by the algorithms will be measured using the \texttt{tracemalloc} library, while the time taken by the algorithms will be measured using the \texttt{time} library.

To ensure a fair comparison, both algorithms will be tested on the same set of input strings and on the same machine, and the results will be averaged over multiple runs to account for any variability in performance.
Additionally, the median, standard deviation, and interquartile range of the results will be calculated to provide a more comprehensive analysis of the performance of both algorithms and to account for the inherent non-determinism of the operating environment.

These results will be compared against the expected theoretical time and space complexities for constructing and querying the \acrshort{sam} and the \acrshort{esa} algorithms.

\clearpage
\section*{Planned Structure}
\addcontentsline{toc}{section}{Planned Structure}
The paper will likely be structured as follows:
\begin{itemize}
    \item Introduction
    \item Research Question and Objectives
    \item 
    \item Literature Review
    \item Methodology
    \item Results and Discussion
    \item Limitations
    \item Conclusion and Future Work
    \item References
    \item AI Declaration
    \item Declaration of Authorship
\end{itemize}

\clearpage

% Back matter in roman numerals
\pagenumbering{roman}

\nocite{*}
\printbibliography

\end{document}
